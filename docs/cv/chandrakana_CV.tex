%%%%%%%%%%%%%%%%%%%%%%%%%%%%%%%%%%%%%%%%%
% Medium Length Graduate Curriculum Vitae
% LaTeX Template
% Version 1.1 (9/12/12)
%
% This template has been downloaded from:
% http://www.LaTeXTemplates.com
%
% Original author:
% Rensselaer Polytechnic Institute (http://www.rpi.edu/dept/arc/training/latex/resumes/)
%
% Important note:
% This template requires the res.cls file to be in the same directory as the
% .tex file. The res.cls file provides the resume style used for structuring the
% document.
%
%%%%%%%%%%%%%%%%%%%%%%%%%%%%%%%%%%%%%%%%%

%----------------------------------------------------------------------------------------
%	PACKAGES AND OTHER DOCUMENT CONFIGURATIONS
%----------------------------------------------------------------------------------------

\documentclass[margin, 10pt]{res} % Use the res.cls style, the font size can be changed to 11pt or 12pt here

\usepackage{helvet} % Default font is the helvetica postscript font
%\usepackage{newcent} % To change the default font to the new century schoolbook postscript font uncomment this line and comment the one above
\usepackage{hyperref}

\usepackage{enumitem}

\setlength{\textwidth}{5.1in} % Text width of the document

\begin{document}

%----------------------------------------------------------------------------------------
%	NAME AND ADDRESS SECTION
%----------------------------------------------------------------------------------------

\moveleft.5\hoffset\centerline{\LARGE\bf Chandrakana Nandi} % Your name at the top

\moveleft\hoffset\vbox{\hrule width\resumewidth height 1pt}\smallskip % Horizontal line after name; adjust line thickness by changing the '1pt'

%\moveleft.5\hoffset\centerline {Paul G. Allen School of Computer Science \& Engineering, University of Washington}
%%\moveleft.5\hoffset\centerline {Box 352350, 185 Stevens Way, Seattle WA 98195-2350}
%\moveleft.5\hoffset\centerline {Website: homes.cs.washington.edu/$\sim$cnandi}
%\moveleft.5\hoffset\centerline {Email: cnandi@cs.washington.edu}


%\moveleft.5\hoffset\centerline {Box 352350, 185 Stevens Way, Seattle WA 98195-2350}

%----------------------------------------------------------------------------------------

\begin{resume}
%\section{Research \\ Interests}

%I am interested in developing techniques which make it easier for programmers to write correct code and using those techniques to build real systems. Currently, I am advised by Dan Grossman in the PLSE group at UW, Seattle.

\section{Employment}

Principal Researcher at Certora Inc., April 2023 - present \\
Affiliate Assistant Professor, UW, Seattle, January 2023 - present \\

Formal Verification Team Lead at Certora Inc., June 2022 - March 2023 \\
Senior Researcher at Certora Inc., September 2021 - May 2022 \\
Website: \url{https://cnandi.com/} \\

\section{Education}

PhD in Computer Science, Fall 2015 - Summer 2021 \\
Paul G. Allen School of Computer Science \& Engineering, UW \\
Thesis topic: Programming Language Techniques for Computational Fabrication \\
Advisors: Zachary Tatlock, Dan Grossman\\
\textbf{Adobe Research Fellow, 2019} \\
%\textbf{Work funded by NSF grant awards: 1813166, 174957} \\

{\sl Master of Science,} Computer Science, June 2018 \\
Paul G. Allen School of Computer Science \& Engineering, UW \\
Advisors: Zachary Tatlock, Dan Grossman

{\sl Master of Science,} Computer Science, August 2014 \\
\'{E}cole Polytechnique F\'{e}d\'{e}rale de Lausanne(EPFL), Switzerland \\
Thesis: Contracts for Real-Time, Safety Critical Systems \\Supervisors: Prof.Viktor Kuncak, Dr.Manuel Oriol\\
\textbf{M.Sc Research Scholars Program and Swiss Government Fellow}

{\sl Bachelor of Science,} Statistics, Mathematics and Computer Science, June 2012 \\
Banaras Hindu University (BHU), Varanasi, India \\
Concentration: Statistics \\
Thesis: Social Network-based Analysis of Behavior \\ Supervisor: Prof. R.D Singh\\
\textbf{University Gold Medalist for graduating with highest GPA}

\section{Publications}
\subsection{Drafts and under submission papers}
%There are 4 top international conferences in the area of programming languages:
%POPL, PLDI, OOPSLA, ICFP.
%Paper acceptance is highly selective, and
%  Distinguished Paper Awards are given to only top 10\% of the accepted papers.
%SIGGRAPH and SIGGRAPH Asia are the top two venues for computer graphics ---
%  the proceedings of both are published in the ACM TOG (Transaction on Graphics).
%UIST is one of the top venues for Human Computer Interaction, and
%ARITH is a top tier venue for computer numerics. \\
\begin{enumerate}
  \item Enzo Nicourt, \textbf{Chandrakana Nandi}, Ylies Falcone. \\
    Towards Mutation-guided Test Suites for Smart Contracts.

  \item Yihong Zhang, Anjali Pal, Adriana Schulz, Zachary Tatlock, \textbf{Chandrakana Nandi}. \\
    Using Anti-Unification to Scale Parametric CAD Decompilation.

  \item John Toman, Shelly Grossman, Alexander Bakst, Mooly Sagiv, \textbf{Chandrakana Nandi}. \\
    Pointer Analysis Guided Verification of Ethereum Bytecode.
\end{enumerate}

\subsection{Peer reviewed conference and workshop papers}
\begin{enumerate}

  \item Anjali Pal, Brett Saiki, Cynthia Richey, Amy Zhu, Ryan Tjoa, Oliver Flatt, Max Willsey, Zachary Tatlock, \textbf{Chandrakana Nandi}. \\
    Equality Saturation Theory Exploration à la Carte, OOPSLA 2023.

  \item David Cao, Rose Kunkel, \textbf{Chandrakana Nandi}, Max Willsey, Zachary Tatlock, Nadia Polikarpova. \\
    \textit{babble}: Learning Better Abstractions with E-Graphs and Anti-Unification, POPL 2023.

  \item Haisen  Zhao, Max Willsey, Amy Zhu, \textbf{Chandrakana Nandi}, Zachary Tatlock, Justin Solomon, Adriana Schulz. \\
    Co-Optimization of Design and Fabrication Plans for Carpentry, SIGGRAPH 2022 (ACM TOG).

  \item \textbf{Chandrakana Nandi}, Max Willsey, Amy zhu, Brett Saiki, Yisu Wang, Adam Anderson, Adriana Schulz, Dan Grossman, Zachary Tatlock.
    Rewrite Rule Inference Using Equality Saturation. OOPSLA 2021. \\
    \textbf{Distinguished Paper Award}

  \item Jasper Tran O'Leary, \textbf{Chandrakana Nandi}, Khang Lee, Nadya Peek. \\
  Taxon: a Language for Formal Reasoning with Digital Fabrication Machines. UIST 2021.

\item Brett Saiki, Oliver Flatt, \textbf{Chandrakana Nandi}, Zachary Tatlock, Pavel Panchekha. \\
  Combining Precision Tuning and Rewriting. ARITH 2021.

\item Molly Carton, \textbf{Chandrakana Nandi}, Adam Anderson, Haisen Zhao, Eva Darulova, Dan Grossman, Jeff Lipton, Adriana Schulz, Zachary Tatlock. \\
  A Roadmap Towards Parallel Printing for Desktop 3D Printers. SFF 2021.

\item Max Willsey, \textbf{Chandrakana Nandi}, Remy Wang, Oliver Flatt, Pavel Panchekha, Zachary Tatlock. \\
    Fast and Extensible Equality Saturation. POPL 2021 \\
    \textbf{Distinguished Paper Award}

  \item \textbf{Chandrakana Nandi}, Max Willsey, Adam Anderson, James R. Wilcox, Eva Darulova, Dan Grossman, Zachary Tatlock. \\
    Synthesizing Structured CAD Models with Equality Saturation and Inverse Transformations. PLDI 2020.
  \item Chenming Wu, Haisen Zhao, \textbf{Chandrakana Nandi}, Jeff Lipton, Zachary Tatlock, Adriana Schulz. \\
    Carpentry Compiler. SIGGRAPH ASIA 2019

  \item \textbf{Chandrakana Nandi}, James R. Wilcox, Pavel Panchekha, Taylor Blau, Dan Grossman, Zachary Tatlock. \\
    Functional Programming for Compiling and Decompiling Computer-aided Design. ICFP 2018.
  \item \textbf{Chandrakana Nandi}, Anat Caspi, Dan Grossman, Zachary Tatlock. \\
    Programming Language Tools and Techniques for 3D Printing. SNAPL 2017.

  \item \textbf{Chandrakana Nandi}, Dan Grossman, Adrian Sampson, Todd Mytkowicz, Kathryn S. McKinley. \\ Debugging Probabilistic Programs. MAPL 2017.
  \item \textbf{Chandrakana Nandi}, Dan Grossman, Adrian Sampson, Todd Mytkowicz, Kathryn S. McKinley. \\ Debugging Probabilistic Programs. PPS 2017.
\item \textbf{Chandrakana Nandi}, Michael D. Ernst. \\ Automatic Trigger Generation for Rule-based Smart Homes. ACM SIGPLAN PLAS 2016.
\item \textbf{Chandrakana Nandi}. Automatic Trigger Generation for End User Written Rules for Home Automation. ACM FSE SRC 2016.
\item \textbf{Chandrakana Nandi}. Correctness and Security for Home Automation. POPL SRC 2016.
\item \textbf{Chandrakana Nandi}, Aurelien Monot, Manuel Oriol: Stochastic Contracts for Runtime Checking of Component-based Real-time Systems. CBSE'15: 18th International ACM SIGSOFT Symposium on Component-Based Software Engineering.
\end{enumerate}
\subsection{Theses}
\begin{enumerate}
\item C. Nandi: Programming Language Tools and Techniques for Computational Fabrication. {PhD Thesis}, UW, August 2021
\item C. Nandi: Functional Programming for Compiling and Decompiling Computer-aided Design, {MS Thesis}, UW, March 2018
\item C. Nandi: Contracts for Real-Time, Safety Critical Systems, {Masters Thesis}, EPFL, August 2014
\item C. Nandi: Social Network based Analysis of Behavior, {Bachelors Thesis}, BHU, April 2012
\end{enumerate}

\section{Professional services}


\subsection{Review committees}
%Top international
%conferences in the field of
%programming languages are PLDI, POPL, OOPSLA, and IFCP. Membership in all the
%following review committees are by invitation only. \\

\begin{enumerate}[itemsep=-2pt]
\item PLDI 2024 program committee (PC)
\item ASPLOS 2024 program committee (PC)
\item TFP 2024 program committee (PC)
\item DSS 2024 review committee
\item ISSTA 2023 program committee (PC)
\item APLAS 2023 program committee (PC)
\item ACM-SRC 2023 Grand Finale judge
\item OOPSLA 2023 external review committee (ERC)
\item SCF 2023 subreviewer
\item TrustX 2023 program committee (PC)
\item EGRAPHS 2023 review committee (PC)
\item PLDI 2022 review committee (PC)
\item OOPSLA 2022 external review committee (ERC)
\item EGRAPHS 2022 review committee (PC)
\item PLDI-SRC 2022 review committee (PC)
\item SFF 2021 review committee
\item Subreviewer, ACM Symposium on Computational Fabrication (SCF), 2020
\item Artifact Evaluation Committee, ASPLOS 2020
\item External Reviewer, ICFP SRC 2019
\item External Reviewer, UBICOMP 2018
\item Reviewer for Elsevier journal, Future Generation Computer Systems 2016
\end{enumerate}

\subsection{Organizing}
\begin{enumerate}[itemsep=-2pt]
  \item ICFP 2025 workshop co-chair
  \item ICFP 2024 workshop co-chair
  \item PLDI-SRC 2024 co-chair
  \item ISSTA-SRC 2023 inaugural co-chair
  \item PLDI-SRC 2023 co-chair
  \item DeFi Security Summit (DSS) Steering committee (current)
  \item National Conference on HPCA, BHU, 2010
\end{enumerate}

\subsection{Other service}
\begin{enumerate}[itemsep=-2pt]
\item {Panelist, Formal Verification and Security panel, ETH Denver 2023}
\item SIGPLAN-M mentor, 2022 - current
\item Session Chair at PLDI 2022 and SIGPLAN tracks
\item Session Chair at EGRAPHS 2022
\item Session Chair at OOPSLA 2021
\item PLMW Panelist, PLDI 2020
\item PSC Chair, Visit Days 2017, UW CSE
\item Student volunteer at POPL 2016
\item ACM-W mentorship program 2015, UW CSE
\item Session Chair at ACM CBSE 2015, session: Component and Composition
\end{enumerate}

\section{Awards}
\begin{enumerate}[itemsep=.5pt]
\item Rising Stars in EECS Award, 2021
\item OOPSLA 2021 \textbf{Distinguished Paper Award}
\item POPL 2021 \textbf{Distinguished Paper Award}
\item Adobe Research Fellowship 2019
\item CRA-W Grad Cohort Workshop 2016 invitee
\item Scholarship and travel grant to attend SIGPLAN PLMW and POPL 2016
\item Student Travel Grants from ACM CCS and PLAS 2016
\item Swiss Government Fellow from September 2012-2014
  % (full fellowship during masters studies at EPFL)
\item M.Sc Research Scholars Program under Prof. Joseph Sifakis, 2012
\item University Gold Medal for having the highest GPA at the
  Institute of Science, BHU, 2012
\item Department Gold Medal for having the highest GPA in the Department of Statistics, BHU, 2012
\item Gargi Devi Deodhar Silver Medal, Institute of Science, BHU, 2012
\item Manorama Gold Medal for having the highest GPA among all Female students at the
  Institute of Science, BHU, 2012
\item Dr. Basudeo Sahni Medal for highest GPA at the
  Institute of Science, BHU, 2012
\item University scholarship holder for academic excellence at the Institute of Science, BHU,
  for 3 consecutive years.
%\begin{itemize}\itemsep -2pt
%\item University Gold Medal for having the highest GPA in the
%  Institute of Science, BHU
%\item Department Gold Medal for having the highest GPA in the Department of Statistics, BHU
%\item Gargi Devi Deodhar Silver Medal for having the highest GPA
%  Institute of Science, BHU
%\item Manorama Gold Medal for having the highest GPA among all Female students in the
%  Institute of Science, BHU
%\item Dr. Basudeo Sahni Medal for highest GPA in the
%  Institute of Science, BHU
%\item Cash award and university scholarship holder for academic excellence
%  for 3 consecutive years
%\end{itemize}
\item Secured All India Rank 14 in the IIT-Joint Admission Test for Mathematical Statistics in 2012
\end{enumerate}

\section{Talks}
 \begin{itemize}[itemsep=.7pt]
 \item{\sl Formal Verification Using the Certora Prover}, Invited Speaker, CMU Secure Blockchain summit, May 2023
 \item{\sl Using Mutation Testing to Write Better Formal Specs}, ETH Denver, March 2023
 \item{\sl Formally verifying what you execute} DeFi FV in Practice, Denver, March 2023
 \item{\sl Formal Verification Using the Certora Prover}, Invited Lecture, UPenn, Nov 2022
 \item{\sl Future of Smart Contract Security}, Panelist, Devcon, Bogota, Oct 2022
 \item {\sl Formal Verification Using the Certora Prover}, Invited Lecture,
   (\href{https://www.certora.com/events/certora-prover-2-day-workshop/}{link})
   Stanford, Aug 2022
 \item {\sl Formal Verification Using the Certora Prover}, Invited Lecture,
   (\href{https://github.com/Certora/Tutorials/tree/master/05.Lesson_GettingFamiliarWithCVT}{link})
   Secureum Bootcamp, Feb 2022
 \item {\sl Rewrite Rule Inference Using Equality Saturation}, Invited Talk, NVIDIA Research, Feb 2022
 \item {\sl Rewrite Rule Inference Using Equality Saturation}, Invited Talk, NUS, Singapore, Nov 2021
 \item {\sl Rewrite Rule Inference Using Equality Saturation}, Conference Talk, OOPSLA, October 2021
 \item {\sl Programming Languages Tools and Techniques for 3D Printing}, Invited Talk, MIT, October 2021
 \item {\sl Efficient Term Rewriting and Rewrite Rule Inference with Equality Saturation}, Invited Talk, Intel, May 2021
 \item {\sl Synthesizing Structured CAD Models with Equality Saturation and Inverse Transformations}, Invited Talk, UCSD PL Seminar, Nov 2020
 \item {\sl Synthesizing Structured CAD Models with Equality Saturation and Inverse Transformations}, Conference Talk, PLDI, June 2020
 \item {\sl Programming Languages for Computational Geometry and Fabrication}, Expert Speaker, IEEE RAS, UP Section Chapter, Allahabad, India, January 2020
 \item {\sl Programming Languages for Computational Fabrication}, Adobe Headquarters, San Jose, August 2019
 \item {\sl Functional Programming for Compiling and Decompiling CAD}, Invited Talk, MPI-SWS, Germany, October 2018
 \item {\sl Functional Programming for Compiling and Decompiling CAD}, Invited Talk, SUNY Buffalo, October 2018
 \item {\sl Functional Programming for Compiling and Decompiling CAD}, Conference  Talk, ICFP, September 2018
 \item {\sl Automatic Trigger Generation for Rule-based Smart Homes}, Conference Talk, PLAS, October 2016
 \item {\sl Debugging Probabilistic Programs}, Internship Talk, Microsoft Research, Redmond, September 2016
 \item {\sl Use of Contracts for Run-time Verification of Real-time Software}, Invited talk, IEEE RAS, UP Section Chapter, Allahabad, India, August 2015
 \item {\sl Stochastic Contracts for Runtime Checking of Component-based Real-time Systems}, Conference talk, ACM Sigsoft Symposium on CBSE, Montreal, May 2015
 \item {\sl Contracts for Real-time and Safety Critical Systems}, LARA, EPFL, August 2014
 \item {\sl Contracts for Real-time and Safety Critical Systems}, ABB Corporate Research, August 2014
 \item {\sl A Bi-directional Model Transformation Tool between BIP and FASA}, ABB Corporate Research, November 2013
 \end{itemize}

\section{Media Coverage}
\begin{itemize}
  \item Featured in the ICFP 2021 trailer
  \item Invited SIGPLAN Article on Fast and Extensible Equality Saturation, April 2021
  \item Co-author of a SIGPLAN blog on Conferences after COVID: An Early-Career Perspective, March 2021
  \item Media coverage for Carpentry Compiler published in 2019:
    \begin{enumerate}
      \item UW News Article, 2019
      \item Tech Crunch Article, 2019
      \item The Daily of UW Article, 2020
    \end{enumerate}
  \item ``I Am CSE" video for Reincarnate, UW CSE 2020
 \end{itemize}

\section{Research Mentorship}
\begin{enumerate}
  \item Ben Kushigian, intern at Certora (March - Sept 2023)
  \item Thia Richey, intern at Certora (June - Sept 2023)
  \item Oliver Flatt, intern at Certora (Jan - Aug 2022)
  \item Vishal Canumalla, intern at Certora (March - June 2022)
  \item Ben Phipathananunth, intern at Certora (June - Dec 2022) $\rightarrow$ Yale PhD student
     \begin{itemize}
       \item $2^{nd}$ place, ACM-SPLASH SRC (undergraduate category), 2022
     \end{itemize}
  \item Amy Zhu, UW PhD Student
  \item Anjali Pal, UW PhD Student
  \item Brett Saiki, UW BS/MS
  \item Anagha G, BIT Hyderabad BS, SIGPLAN-M, Jan 2023 - current
  \item Alperen Keles, UMD PhD student, SIGPLAN-M, May 2022 - Jan 2023
  \item Adam Anderson, UW BS (March 2018 - Nov 2020)
  \item Grace Oh, High School $\rightarrow$ Princeton BS
  \item Taylor Blau, UW BS $\rightarrow$ Github
  \item Seth Pendergrass, UW BS $\rightarrow$ Microsoft
  \item Melissa Hovik, UW BS/MS $\rightarrow$ Caltech, Teaching Faculty
\end{enumerate}

\section{TA Experience}
\begin{enumerate}
\item Winter 2018. CSE 341: Programming Languages, UW, Seattle
\item Spring, Fall 2016. CSE 331: Software Design and Implementation, UW, Seattle
\item Spring 2015: Concepts of Concurrent Computation, graduate level, ETH Zurich
\item Fall 2014: Introduction to Programming, undergraduate level, ETH Zurich
\end{enumerate}



\section{Prior Work \\ Experience}
{\sl Research Intern} \hfill June 2016 - August 2016 \\
RiSE group, Microsoft Research, Redmond
%\begin{itemize} \itemsep -2pt
%    \item Developed a probabilistic programming language to make statistical
%      inference a first class citizen.
%    \item Designed and implemented a debugger for probabilistic programs to work
%      with the language.
%    \item Evaluated the debugger on a variety of benchmarks demonstrating
%      the benefits of the debugger and the ability to express statistical
%      inference in the language.
%    \item Wrote and published a research paper at an ACM symposium (MAPL 2017)
%      and a workshop (PPS 2017).
%
%\end{itemize}

{\sl Research Assistant} \hfill September 2014 - June 2015 \\
Chair of Software Engineering, ETH Zurich
%\begin{itemize} \itemsep -2pt % Reduce space between items
%\item Worked on specifying functional properties of concurrent OO programs
%  in the SCOOP concurrency model in the programming language Eiffel.
%\item This was useful for verifying the correctness of concurrent programs,
%  an incredibly challenging problem in programming language and formal methods.
%\item While at ETH, I also TA-ed for a graduate level class on concurrent programming,
%  "Concepts of Concurrent Computation". I developed exciting
%    verification assignments using model checking and Hoare Logic as part of my
%    TA work.
%\end{itemize}

{\sl Research Intern} \hfill August 2013 - August 2014 \\
ABB Corporate Research Center, Switzerland
%\begin{itemize}
%  \item Designed and implemented a bi-directional model transformer
%    and compiler between two component based frameworks used for writing and
%    verifying real-time safety-critical software.
%  \item My internship was extended by another 6 months based on the success of
%    my first project.
%    I started a second project in collaboration with
%    Prof. Viktor Kuncak's lab at
%    EPFL University (Switzerland) that focused on formal verification of software.
%\item For this project, I
%  designed and implemented a framework for formally
%    specifying and checking
%    functional and real-time properties of component-based industrial software.
%\item This project also contributed to my MS thesis work at EPFL and led
%  to a publication at an international ACM Conference (CBSE 2015).
%   % \href{http://lara.epfl.ch/~kuncak/}{Prof. Viktor Kuncak},
%   % \href{http://www-users.cs.york.ac.uk/~manuel/Manuel_Oriols_Website/Profile.html} {Dr. Manuel Oriol}
%\end{itemize}

{\sl Research Intern}  \hfill June 2010 - July 2010 \\
Bio-robotics Lab, EPFL, Switzerland
%\begin{itemize}
%\item Analyzed the locomotion of a salamander from X-Ray movies and
%  implemented a graphical simulation of the temporal variations in the angles
%  at the different joints on the salamander's body
%\item This was very useful for the design of the famous salamander robot
%  developed at EPFL called \href{http://biorob.epfl.ch/salamandra}{Salamandra Robotica}.
%\end{itemize}


% \section{Technical Skills}
% \begin{itemize}
% \item \textbf{Programming languages:} Python, Java, C++, Racket, Coq, JavaScript, C, PHP(basics), Eiffel
% \item \textbf{Markup languages, DSLs:} HTML, CSS, XML, SQL, Matlab
% \item \textbf{Others:} Latex, Git, Ogre3D, Bullet physics
% \item \textbf{OS:} Linux, Windows
% \end{itemize}

% \section{Events}
% \begin{itemize}[itemsep=.7pt]
% \item {Forum for Women in Computing, Department of Computer Science, May 2015, ETH Zurich}
% \item {Workshop on Software Correctness and Reliability, October 3-4, 2014, ETH Zurich }
% \item {FMCAD October 21-24, 2014, EPFL Lausanne}
% \end{itemize}


\section{Language Proficiency}
English: Fluent\\
French: Basic\\
German: Basic\\
Bengali: Mother Tongue\\
Hindi: Fluent

\end{resume}
\end{document}
